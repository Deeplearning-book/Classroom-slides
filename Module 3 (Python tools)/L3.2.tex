\documentclass{beamer}
\setbeamertemplate{footline}[page number]
\date{}
\author{}
\institute{}

%%%%%%% Put these names back in the final version 
%\\Aswathy Rajendra Kurup\\Meenu Ajith}
%\institute{Department of Electrical and Computer Engineering\\The University of New Mexico}
\setbeamercovered{transparent}
\usepackage{setspace}
\usepackage{array}
\usepackage[T1]{fontenc}
\usepackage{graphicx}
\usepackage{amsmath}
\usepackage{amsfonts}
\usepackage{amssymb}
\usepackage{makeidx}
\usefonttheme{serif}
\usepackage{multirow}
\usepackage{booktabs} 
\usepackage{rotating}
\usepackage{color}
\usepackage{float}
\usepackage[latin1]{inputenc}
\usepackage[english]{babel}
\usepackage{amsmath}
\usepackage{amsfonts}
\usepackage{eurosym}
\usepackage{rotating}
\usepackage{multicol}
\usepackage{pythonhighlight}
\usepackage[normalem]{ulem}
\newcommand{\ba}{{\bf a}}
\newcommand{\bb}{{\bf b}}
\newcommand{\bc}{{\bf c}}
\newcommand{\bd}{{\bf d}}
\newcommand{\be}{{\bf e}}
\newcommand{\bbf}{{\bf f}}
\newcommand{\bg}{{\bf g}}
\newcommand{\bh}{{\bf h}}
\newcommand{\bi}{{\bf i}}
\newcommand{\bk}{{\bf k}}
\newcommand{\bl}{{\bf l}}
\newcommand{\bm}{{\bf m}}
\newcommand{\bn}{{\bf n}}
\newcommand{\bo}{{\bf o}}
\newcommand{\bp}{{\bf p}}
\newcommand{\bq}{{\bf q}}
\newcommand{\br}{{\bf r}}
\newcommand{\bs}{{\bf s}}
\newcommand{\bt}{{\bf t}}
\newcommand{\bu}{{\bf u}}
\newcommand{\bv}{{\bf v}}
\newcommand{\bw}{{\bf w}}
\newcommand{\bx}{{\bf x}}
\newcommand{\by}{{\bf y}}
\newcommand{\bz}{{\bf z}}

\newcommand{\bA}{{\bf A}}
\newcommand{\bB}{{\bf B}}
\newcommand{\bC}{{\bf C}}
\newcommand{\bE}{{\bf E}}
\newcommand{\bG}{{\bf G}}
\newcommand{\bH}{{\bf H}}
\newcommand{\bI}{{\bf I}}
\newcommand{\bK}{{\bf K}}
\newcommand{\bL}{{\bf L}}
\newcommand{\bM}{{\bf M}}
\newcommand{\bO}{{\bf O}}
\newcommand{\bQ}{{\bf Q}}
\newcommand{\bR}{{\bf R}}
\newcommand{\bS}{{\bf S}}
\newcommand{\bT}{{\bf T}}
\newcommand{\bV}{{\bf V}}
\newcommand{\bW}{{\bf W}}
\newcommand{\bX}{{\bf X}}
\newcommand{\bY}{{\bf Y}}
\newcommand{\bZ}{{\bf Z}}
\newcommand\uptocnt{\stackrel{\mathclap{\normalfont\mbox{c}}}{\propto}}
\newcommand{\bpt}{{\bf pt}}
\newcommand{\bpl}{{\bf pl}}
\newcommand{\bdp}{{\bf dp}}
\newcommand{\btemp}{{\bf temp}}

\newcommand{\bmu}{{\boldsymbol \mu}}
\newcommand{\bSigma}{{\boldsymbol \Sigma}}
\newcommand{\bsigma}{{\boldsymbol \sigma}}
\newcommand{\bvarPhi}{{\boldsymbol \varPhi}}
\newcommand{\bvarphi}{{\boldsymbol \varphi}}
\newcommand{\bPhi}{{\boldsymbol \Phi}}
\newcommand{\bdelta}{{\boldsymbol \delta}}
\newcommand{\bZero}{{\bf 0}}
\newcommand{\bOne}{{\bf 1}}
\newcommand{\balpha}{{\boldsymbol \alpha}}
\newcommand{\bAlpha}{{\boldsymbol A}}
\newcommand{\btheta}{{\boldsymbol \theta}}

\newcommand{\softmax}{\text{softmax}}
\newcommand{\diag}{\text{diag}}
\newcommand{\sinc}{\mathrm{sinc}}
\newcommand{\argmin}{\mathop{\mathrm{argmin}}}
\newcommand{\infl}{\eta}
\newcommand{\Ind}{\mathrm{I}}
\newcommand{\Real}{\mathbb R}
\newcommand{\Intg}{\mathbb Z}
\newcommand{\Complex}{\mathbb C}
\newcommand{\Natural}{\mathbb N}
\newcommand{\Fourier}[1]{\mathcal{F} \{#1\}}
%\newcommand{\ii}{\mathbbm{i}}
\newcommand{\bphi}{\boldsymbol{\mathit{\phi}}}

\newcommand{\hs}{\hspace{2pt}}
\newcommand{\sign}{\text{sign}}
\author{Manel Mart\'inez-Ram\'on\\Meenu Ajith\\Aswathy Rajendra Kurup}

\usetheme{Madrid}
\usecolortheme{beaver}
\usepackage{tikz}
\usetikzlibrary{fit,arrows,calc,positioning}
\usepackage{listings}
\usepackage{xcolor}
\usepackage{emerald} 
\usepackage[T1]{fontenc} 
\usepackage{verbatim}
\usepackage{graphicx}
\usepackage{epsfig}
\usepackage{psfrag}
\usepackage[english]{babel}
\usepackage{listings}
\usepackage{courier}
\usepackage{color}
 \usepackage{vwcol} 
 \usepackage[english]{babel} % To obtain English text with the blindtext package
\usepackage{blindtext}
\definecolor{codegreen}{rgb}{0,0.6,0}
\definecolor{codegray}{rgb}{0.5,0.5,0.5}
\definecolor{codepurple}{rgb}{0.58,0,0.82}
\definecolor{backcolour}{rgb}{0.95,0.95,0.92}

\lstdefinestyle{mystyle}{
  backgroundcolor=\color{backcolour},   commentstyle=\color{codegreen},
  keywordstyle=\color{magenta},
  numberstyle=\tiny\color{codegray},
  stringstyle=\color{codepurple},
  basicstyle=\ttfamily\footnotesize,
  breakatwhitespace=false,         
  breaklines=true,                 
  captionpos=b,                    
  keepspaces=true,                 
  numbers=left,                    
  numbersep=5pt,                  
  showspaces=false,                
  showstringspaces=false,
  showtabs=false,                  
  tabsize=2
}
\lstset{style=mystyle}

%% Stuff for movies

% %\newcommand{\bt}{{\bf t}}
% \newcommand{\br}{{\bf r}}
% \newcommand{\bs}{{\bf s}}
% \newcommand{\by}{{\bf y}}
% \newcommand{\bz}{{\bf z}}
% \newcommand{\bx}{{\bf x}}
% \newcommand{\bw}{{\bf w}}
% \newcommand{\be}{{\bf e}}
% \newcommand{\bbf}{{\bf f}}
% \newcommand{\bb}{{\bf b}}
% \newcommand{\bd}{{\bf d}}
% \newcommand{\bA}{{\bf A}}
% \newcommand{\bB}{{\bf B}}
% \newcommand{\bL}{{\bf L}}
% \newcommand{\bM}{{\bf M}}

% \newcommand{\bC}{{\bf C}}
% \newcommand{\bI}{{\bf I}}
% \newcommand{\bK}{{\bf K}}
% \newcommand{\bk}{{\bf k}}
% \newcommand{\bT}{{\bf T}}
% \newcommand{\bV}{{\bf V}}
% \newcommand{\bW}{{\bf W}}
% \newcommand{\bX}{{\bf X}}
% \newcommand{\bY}{{\bf Y}}
% \newcommand{\bZ}{{\bf Z}}
% \newcommand{\bm}{{\bf m}}
% \newcommand{\bpt}{{\bf pt}}
% \newcommand{\bpl}{{\bf pl}}
% \newcommand{\bdp}{{\bf dp}}
% \newcommand{\btemp}{{\bf temp}}
% \newcommand{\bl}{{\bf l}}
% \newcommand{\bu}{{\bf u}}
% \newcommand{\bmu}{{\boldsymbol \mu}}
% \newcommand{\bSigma}{{\boldsymbol \Sigma}}
% \newcommand{\bLambda}{{\boldsymbol \Lambda}}

% \newcommand{\bsigma}{{\boldsymbol \sigma}}
% \newcommand{\bvarphi}{{\boldsymbol \varPhi}}
% \newcommand{\btheta}{{\boldsymbol \theta}}
% \newcommand{\bZero}{{\bf 0}}
% \newcommand{\balpha}{{\boldsymbol \alpha}}
% \newcommand{\bpi}{{\boldsymbol \pi}}
% \newcommand{\bxi}{{\boldsymbol \xi}}
% \newcommand{\bdelta}{{\boldsymbol \delta}}
\lstset{
	language=Python,
	basicstyle=\footnotesize\ttfamily\color{black},
	commentstyle = \footnotesize\ttfamily\color{red},
	keywordstyle=\footnotesize\ttfamily\color{blue},
	stringstyle=\footnotesize\ttfamily\color{black},
%	columns=fixed,
%	numbers=left,    
	numberstyle=\tiny,
	stepnumber=1,
	numbersep=5pt,
	tabsize=1,
	extendedchars=true,
	breaklines=true,            
	frame=b,         
	showspaces=false,
	showtabs=true,
	xleftmargin=6pt,
	framexleftmargin=6pt,
	framexrightmargin=2pt,
	framexbottommargin=4pt,
	showstringspaces=false      
}

\lstloadlanguages{
         Python
}

%\graphicspath{ {./images/} }  % Figures path - used in graphicx

%\selectcolormodel{cmyk}

\mode<presentation>

\newcommand{\dred}{darkred!90!black}
\newcommand{\written}{\ECFJD\textcolor{cyan!50!white}}
\newcommand{\hlight}{\textcolor{\dred}}
\newcommand{\Ex}{\textcolor{\dred}{Ex. }}

% remove navigation symbols in full screen mode
\setbeamertemplate{navigation symbols}{}  
\setbeamertemplate{blocks}[rounded][shadow=false]
\setbeamercolor{note page}{fg=black}

\setbeamercolor{title}{fg=\dred}
\setbeamercolor{frametitle}{fg=white}
\setbeamercolor{frametitle}{bg=\dred}
\setbeamercolor{structure}{fg=black,bg=white}
\setbeamercolor{background canvas}{bg=white,fg=black}
\setbeamercolor{normal text}{fg=black,bg=white}
\setbeamercolor{item}{fg=red!80!black,bg=white!}
\addtobeamertemplate{block begin}{\setbeamercolor{block title}{fg=white,bg=\dred}
\setbeamercolor{block body}{fg=white,bg=gray}}{}



\title{3. Deep learning tools}
\subtitle{3.2 Numpy}

\addtobeamertemplate{frametitle}{}


\begin{document}

\maketitle

\begin{frame}{Introduction}
\begin{itemize}
\item Standard package for scientific computing. 
\item provides advanced mathematical computations and operations using multi-dimensional arrays and matrices. 
\item The base variable is an multinimensional array (ndarray)
\item This lesson summarizes array initialization,  types of operations, extracting shape, axis properties.
\end{itemize}
\end{frame}

\begin{frame}[fragile]{Arrays}
The numpy array can be initialized with the command  \textbf{np.array}. The array has the following methods:
\begin{itemize}
    \item \textbf{array.shape}: This gives the shape of the array.
\item \textbf{array.size} : This gives the total number of elements in an array.
\item \textbf{array.ndim} : Number of axes in an array.
\item \textbf{array.dtype}: Gives the datatype of elements in the array.
\end{itemize}
\begin{lstlisting}
import numpy as np
my_array = np.array([[1, 2, 3, 4],
                     [5, 6, 7, 8]]) 
#initializing a different data type array
my_array.shape #Ouputs (2,4)
\end{lstlisting}
\end{frame}

\begin{frame}[fragile]{Predefined arrays}
\begin{lstlisting}
mat1 = np.zeros((5,4))      #matrix of zeros
mat2 = np.ones((3,2))       #matrix of ones
mat3 = np.empty((2,2))      #empty matrix
mat4 = np.eye(3)            #3x3 identity matrix
mat5 = np.full((3,3),2)     #matrix of 2's
mat6 = np.arange(1,50,3)    #Sequence of integers
mat7 = np.linspace(1,49,17) #The same sequence but with reals

\end{lstlisting}
Lines 5 and 6 generate a sequence of 17 numbers from 1 to 49.

\end{frame}
\begin{frame}[fragile]{Changing the shape}
\begin{lstlisting}
import numpy as np
ar = np.array([[1,2],[3,4],[5,6],[7,8])
print(ar)
\end{lstlisting}
\begin{tiny}
\begin{verbatim}
 [[1 2]
  [3 4]
  [5 6]
  [7 8]]
\end{verbatim}
\end{tiny}
\begin{lstlisting}
ar1 = ar.reshape((2,4)) #modifying the shape
print(ar1)
\end{lstlisting}
\begin{tiny}
\begin{verbatim}
[[1 2 3 4]
[5 6 7 8]]
\end{verbatim}
\end{tiny}
\begin{lstlisting}
ar2=ar1.ravel()
print(ar2)
\end{lstlisting}
\begin{tiny}
\begin{verbatim}
[1 2 3 4 5 6 7 8]
\end{verbatim}
\end{tiny}
\end{frame}
\begin{frame}[fragile]{Stacking and splitting}
\begin{lstlisting}
array1 = np.array([1,2,3,4]) # initializing two arrays
array2 = np.array([5,6,7,8])
ar_row = np.hstack((array1,array2)) # using hstack
print(ar_row)
\end{lstlisting}
\begin{tiny}
\begin{verbatim}
    [1 2 3 4 5 6 7 8]
\end{verbatim}
\end{tiny}

\begin{lstlisting}
ar_column = np.column_stack((array1,array2)) # using column_stack
print(ar_column)
\end{lstlisting}

\begin{tiny}
\begin{verbatim}
   [[1 5]
    [2 6]
    [3 7]
    [4 8]]
\end{verbatim}
\end{tiny}
\end{frame}

\begin{frame}[fragile]{Stacking and splitting}

\begin{lstlisting}
print(np.hsplit(ar_row,2)) #splitting along column
\end{lstlisting}



\begin{tiny}
\begin{verbatim}
    [array([[1],
           [2],
           [3],
           [4]]),
    array([[5],
           [6],
           [7],
           [8]])]
\end{verbatim}
\end{tiny}


\begin{lstlisting}
print(np.array_split(ar_row,2, axis = 0)) # along axis = 0
\end{lstlisting}

\begin{tiny}
\begin{verbatim}
    [array([[1, 5],
           [2, 6]]), 
     array([[3, 7],
           [4, 8]])]
\end{verbatim}
\end{tiny}

\end{frame}
\begin{frame}[fragile]{Arithmetic operations}
\begin{lstlisting}
a = np.array([1,2,3,4])     #creating arrays a and b
b = np.array([4.,5.,1.,2.]) #b contains reals
add_ab = np.add(a,b)        #addition 
rec_b = np.reciprocal(b)    #reciprocal of array b (inverse)
pow_ab = np.power(a,b)      #power of a to b
sqrt_a = np.sqrt(a)         #square root of a
\end{lstlisting}
\begin{lstlisting}
zip_obj = zip(a, b)         #create a list of tuples
comp = []                   #create an empty list
for x,y in zip_obj:         #obtain each element zup_obj
    c = complex(x,y)        #compute x+jy  
    comp.append(c)          #append these to the empty list
print(list(comp))              # printing complex array
print(list(np.real(comp)))     # getting the real part
print(list(np.imag(comp)))     # getting imag part 
\end{lstlisting}
\begin{tiny}
\begin{verbatim}
    [(1+4j), (2+5j), (3+1j), (4+2j)]
    [1.0, 2.0, 3.0, 4.0]
    [4.0, 5.0, 1.0, 2.0]
\end{verbatim}
\end{tiny}
\end{frame}
\begin{frame}[fragile]{Mathematical functions}
\begin{lstlisting}
a = np.array([30,45,60,90])   #an array of degrees
print(np.sin(np.radians(a)))  #convert to rads, print its sine
b = np.array([0.35066070245,2.67822320434]) #an array of ints
print(np.around(b, 4)) #print their round to the 4th decimal
print(np.floor(b))     # use floor command
print(np.ceil(b))      # use ceil command
\end{lstlisting}
\end{frame}
\end{document}	