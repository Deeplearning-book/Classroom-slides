\documentclass{beamer}
\setbeamertemplate{footline}[page number]
\date{}
\author{}
\institute{}

%%%%%%% Put these names back in the final version 
%\\Aswathy Rajendra Kurup\\Meenu Ajith}
%\institute{Department of Electrical and Computer Engineering\\The University of New Mexico}
\setbeamercovered{transparent}
\usepackage{setspace}
\usepackage{array}
\usepackage[T1]{fontenc}
\usepackage{graphicx}
\usepackage{amsmath}
\usepackage{amsfonts}
\usepackage{amssymb}
\usepackage{makeidx}
\usefonttheme{serif}
\usepackage{multirow}
\usepackage{booktabs} 
\usepackage{rotating}
\usepackage{color}
\usepackage{float}
\usepackage[latin1]{inputenc}
\usepackage[english]{babel}
\usepackage{amsmath}
\usepackage{amsfonts}
\usepackage{eurosym}
\usepackage{rotating}
\usepackage{multicol}
\usepackage{pythonhighlight}
\usepackage[normalem]{ulem}
\newcommand{\ba}{{\bf a}}
\newcommand{\bb}{{\bf b}}
\newcommand{\bc}{{\bf c}}
\newcommand{\bd}{{\bf d}}
\newcommand{\be}{{\bf e}}
\newcommand{\bbf}{{\bf f}}
\newcommand{\bg}{{\bf g}}
\newcommand{\bh}{{\bf h}}
\newcommand{\bi}{{\bf i}}
\newcommand{\bk}{{\bf k}}
\newcommand{\bl}{{\bf l}}
\newcommand{\bm}{{\bf m}}
\newcommand{\bn}{{\bf n}}
\newcommand{\bo}{{\bf o}}
\newcommand{\bp}{{\bf p}}
\newcommand{\bq}{{\bf q}}
\newcommand{\br}{{\bf r}}
\newcommand{\bs}{{\bf s}}
\newcommand{\bt}{{\bf t}}
\newcommand{\bu}{{\bf u}}
\newcommand{\bv}{{\bf v}}
\newcommand{\bw}{{\bf w}}
\newcommand{\bx}{{\bf x}}
\newcommand{\by}{{\bf y}}
\newcommand{\bz}{{\bf z}}

\newcommand{\bA}{{\bf A}}
\newcommand{\bB}{{\bf B}}
\newcommand{\bC}{{\bf C}}
\newcommand{\bE}{{\bf E}}
\newcommand{\bG}{{\bf G}}
\newcommand{\bH}{{\bf H}}
\newcommand{\bI}{{\bf I}}
\newcommand{\bK}{{\bf K}}
\newcommand{\bL}{{\bf L}}
\newcommand{\bM}{{\bf M}}
\newcommand{\bO}{{\bf O}}
\newcommand{\bQ}{{\bf Q}}
\newcommand{\bR}{{\bf R}}
\newcommand{\bS}{{\bf S}}
\newcommand{\bT}{{\bf T}}
\newcommand{\bV}{{\bf V}}
\newcommand{\bW}{{\bf W}}
\newcommand{\bX}{{\bf X}}
\newcommand{\bY}{{\bf Y}}
\newcommand{\bZ}{{\bf Z}}
\newcommand\uptocnt{\stackrel{\mathclap{\normalfont\mbox{c}}}{\propto}}
\newcommand{\bpt}{{\bf pt}}
\newcommand{\bpl}{{\bf pl}}
\newcommand{\bdp}{{\bf dp}}
\newcommand{\btemp}{{\bf temp}}

\newcommand{\bmu}{{\boldsymbol \mu}}
\newcommand{\bSigma}{{\boldsymbol \Sigma}}
\newcommand{\bsigma}{{\boldsymbol \sigma}}
\newcommand{\bvarPhi}{{\boldsymbol \varPhi}}
\newcommand{\bvarphi}{{\boldsymbol \varphi}}
\newcommand{\bPhi}{{\boldsymbol \Phi}}
\newcommand{\bdelta}{{\boldsymbol \delta}}
\newcommand{\bZero}{{\bf 0}}
\newcommand{\bOne}{{\bf 1}}
\newcommand{\balpha}{{\boldsymbol \alpha}}
\newcommand{\bAlpha}{{\boldsymbol A}}
\newcommand{\btheta}{{\boldsymbol \theta}}

\newcommand{\softmax}{\text{softmax}}
\newcommand{\diag}{\text{diag}}
\newcommand{\sinc}{\mathrm{sinc}}
\newcommand{\argmin}{\mathop{\mathrm{argmin}}}
\newcommand{\infl}{\eta}
\newcommand{\Ind}{\mathrm{I}}
\newcommand{\Real}{\mathbb R}
\newcommand{\Intg}{\mathbb Z}
\newcommand{\Complex}{\mathbb C}
\newcommand{\Natural}{\mathbb N}
\newcommand{\Fourier}[1]{\mathcal{F} \{#1\}}
%\newcommand{\ii}{\mathbbm{i}}
\newcommand{\bphi}{\boldsymbol{\mathit{\phi}}}

\newcommand{\hs}{\hspace{2pt}}
\newcommand{\sign}{\text{sign}}
\author{Manel Mart\'inez-Ram\'on\\Meenu Ajith\\Aswathy Rajendra Kurup}

\usetheme{Madrid}
\usecolortheme{beaver}
\usepackage{tikz}
\usetikzlibrary{fit,arrows,calc,positioning}
\usepackage{listings}
\usepackage{xcolor}
\usepackage{emerald} 
\usepackage[T1]{fontenc} 
\usepackage{verbatim}
\usepackage{graphicx}
\usepackage{epsfig}
\usepackage{psfrag}
\usepackage[english]{babel}
\usepackage{listings}
\usepackage{courier}
\usepackage{color}
 \usepackage{vwcol} 
 \usepackage[english]{babel} % To obtain English text with the blindtext package
\usepackage{blindtext}
\definecolor{codegreen}{rgb}{0,0.6,0}
\definecolor{codegray}{rgb}{0.5,0.5,0.5}
\definecolor{codepurple}{rgb}{0.58,0,0.82}
\definecolor{backcolour}{rgb}{0.95,0.95,0.92}

\lstdefinestyle{mystyle}{
  backgroundcolor=\color{backcolour},   commentstyle=\color{codegreen},
  keywordstyle=\color{magenta},
  numberstyle=\tiny\color{codegray},
  stringstyle=\color{codepurple},
  basicstyle=\ttfamily\footnotesize,
  breakatwhitespace=false,         
  breaklines=true,                 
  captionpos=b,                    
  keepspaces=true,                 
  numbers=left,                    
  numbersep=5pt,                  
  showspaces=false,                
  showstringspaces=false,
  showtabs=false,                  
  tabsize=2
}
\lstset{style=mystyle}

%% Stuff for movies

% %\newcommand{\bt}{{\bf t}}
% \newcommand{\br}{{\bf r}}
% \newcommand{\bs}{{\bf s}}
% \newcommand{\by}{{\bf y}}
% \newcommand{\bz}{{\bf z}}
% \newcommand{\bx}{{\bf x}}
% \newcommand{\bw}{{\bf w}}
% \newcommand{\be}{{\bf e}}
% \newcommand{\bbf}{{\bf f}}
% \newcommand{\bb}{{\bf b}}
% \newcommand{\bd}{{\bf d}}
% \newcommand{\bA}{{\bf A}}
% \newcommand{\bB}{{\bf B}}
% \newcommand{\bL}{{\bf L}}
% \newcommand{\bM}{{\bf M}}

% \newcommand{\bC}{{\bf C}}
% \newcommand{\bI}{{\bf I}}
% \newcommand{\bK}{{\bf K}}
% \newcommand{\bk}{{\bf k}}
% \newcommand{\bT}{{\bf T}}
% \newcommand{\bV}{{\bf V}}
% \newcommand{\bW}{{\bf W}}
% \newcommand{\bX}{{\bf X}}
% \newcommand{\bY}{{\bf Y}}
% \newcommand{\bZ}{{\bf Z}}
% \newcommand{\bm}{{\bf m}}
% \newcommand{\bpt}{{\bf pt}}
% \newcommand{\bpl}{{\bf pl}}
% \newcommand{\bdp}{{\bf dp}}
% \newcommand{\btemp}{{\bf temp}}
% \newcommand{\bl}{{\bf l}}
% \newcommand{\bu}{{\bf u}}
% \newcommand{\bmu}{{\boldsymbol \mu}}
% \newcommand{\bSigma}{{\boldsymbol \Sigma}}
% \newcommand{\bLambda}{{\boldsymbol \Lambda}}

% \newcommand{\bsigma}{{\boldsymbol \sigma}}
% \newcommand{\bvarphi}{{\boldsymbol \varPhi}}
% \newcommand{\btheta}{{\boldsymbol \theta}}
% \newcommand{\bZero}{{\bf 0}}
% \newcommand{\balpha}{{\boldsymbol \alpha}}
% \newcommand{\bpi}{{\boldsymbol \pi}}
% \newcommand{\bxi}{{\boldsymbol \xi}}
% \newcommand{\bdelta}{{\boldsymbol \delta}}
\lstset{
	language=Python,
	basicstyle=\footnotesize\ttfamily\color{black},
	commentstyle = \footnotesize\ttfamily\color{red},
	keywordstyle=\footnotesize\ttfamily\color{blue},
	stringstyle=\footnotesize\ttfamily\color{black},
%	columns=fixed,
%	numbers=left,    
	numberstyle=\tiny,
	stepnumber=1,
	numbersep=5pt,
	tabsize=1,
	extendedchars=true,
	breaklines=true,            
	frame=b,         
	showspaces=false,
	showtabs=true,
	xleftmargin=6pt,
	framexleftmargin=6pt,
	framexrightmargin=2pt,
	framexbottommargin=4pt,
	showstringspaces=false      
}

\lstloadlanguages{
         Python
}

%\graphicspath{ {./images/} }  % Figures path - used in graphicx

%\selectcolormodel{cmyk}

\mode<presentation>

\newcommand{\dred}{darkred!90!black}
\newcommand{\written}{\ECFJD\textcolor{cyan!50!white}}
\newcommand{\hlight}{\textcolor{\dred}}
\newcommand{\Ex}{\textcolor{\dred}{Ex. }}

% remove navigation symbols in full screen mode
\setbeamertemplate{navigation symbols}{}  
\setbeamertemplate{blocks}[rounded][shadow=false]
\setbeamercolor{note page}{fg=black}

\setbeamercolor{title}{fg=\dred}
\setbeamercolor{frametitle}{fg=white}
\setbeamercolor{frametitle}{bg=\dred}
\setbeamercolor{structure}{fg=black,bg=white}
\setbeamercolor{background canvas}{bg=white,fg=black}
\setbeamercolor{normal text}{fg=black,bg=white}
\setbeamercolor{item}{fg=red!80!black,bg=white!}
\addtobeamertemplate{block begin}{\setbeamercolor{block title}{fg=white,bg=\dred}
\setbeamercolor{block body}{fg=white,bg=gray}}{}



\title{3. Deep learning tools}
\subtitle{3.9 TensorFlow and Keras}

\addtobeamertemplate{frametitle}{}


\begin{document}

\maketitle

\begin{frame}{Elements of Tensorflow}{Introduction}
\begin{itemize}
\item Open-source machine learning library developed by the Google Brain team in 2012. 
\item It has tools and libraries for developing deep learning applications. 
\item It uses multidimensional arrays (tensors) as the basic elements. 
\item Efficient management  of large-scale datasets.
\item GPU support.
\end{itemize}
\end{frame}

\begin{frame}[fragile]{Elements of Tensorflow}

\begin{table}[H]
\begin{tabular}{l|l|l}
\textbf{Term} &
  \textbf{Definition} &
  \textbf{Examples} \\ \hline
Tensor &
  \begin{tabular}[c]{@{}l@{}}Array containing the data\\ in multiple dimensions.\end{tabular} &
 \\ 
Shape &
  \begin{tabular}[c]{@{}l@{}}Number of elements\\ in each dimension of a \\ tensor.\end{tabular} &
 % \begin{tabular}[c]{@{}l@{}}
\begin{lstlisting}[numbers=none]
Scalar = 25, shape = []
Vector = [1,2], shape = [2]
Matrix = [[3,4],
          [5,5]],
          shape = [2,2]
Tensor = [[[1,2],
           [2,5],
           [3,2]]
          [[4,5],
           [7,8],
           [4,1]]], 
           shape = [2, 3, 2]
\end{lstlisting}
%\end{tabular}
\\
\end{tabular}
\end{table}
\end{frame}


\begin{frame}[fragile]{Elements of Tensorflow}{Terms and definitions}

\begin{table}[H]
\begin{tabular}{l|l|l}
\textbf{Term} &
  \textbf{Definition} &
  \textbf{Examples} \\ \hline
Type &
  \begin{tabular}[c]{@{}l@{}}Data structure to represent \\tensors. \\ \\Common tensor types:\\\\  1. Constant : Fixed values\\\\ 2. Variables : Values can be\\updated over time. \\\\  3. Placeholder: \\ Initialization not required.\end{tabular} &
  \begin{lstlisting}[numbers=none]
tf.constant(5)

tf.Variable([[1,1],[5,5]],
name = 'matrix')

tf.placeholder(tf.float64,
shape = {[}None,5{]})
\end{lstlisting}
 \end{tabular}
 \end{table}
\end{frame}

\begin{frame}[fragile]{Elements of Tensorflow}{Terms and definitions}

\begin{table}[H]
\begin{tabular}{l|l|l}
\textbf{Term} &
  \textbf{Definition} &
  \textbf{Examples} \\ \hline
Graph &
  \begin{tabular}[c]{@{}l@{}}Data structures \\ consisting of nodes \\ that allows the flow of \\ computational operations.\end{tabular} &
  \begin{tabular}[c]{@{}l@{}}a = 10\\ b = 6\\ out = tf.subtract(a,b)\end{tabular} \\ \hline
Session &
  \begin{tabular}[c]{@{}l@{}}Used to evaluate\\ the computational\\ operations in a graph.\end{tabular} &
  \begin{tabular}[c]{@{}l@{}}s = tf.Session()\\ s.run(out)\\ s.close()\end{tabular} \\ \hline
\end{tabular}
\end{table}
Examples are provided in separate notebooks.
\end{frame}

\begin{frame}{Elements of Keras}{Introduction}

\begin{itemize}
    \item TensorFlow is a low-level language with high complexity. 
    \item Keras can be used to simplify these complexities. 
    \item Developed at Google by Francois Chollet.
\item 
High-level deep learning library. It
can run on top of  Tensorflow, Cognitive Toolkit (CNTK), Theano, etc. 
\item Minimalistic structure for 
faster implementation of complex structures. 

\item Supports CPU and
GPU.
\item Widely used in machine learning, computer vision, and
time series-related applications.
\item    The components of Keras are  Models, Layers and Core. 
\end{itemize}
\end{frame}
\begin{frame}{Elements of keras}{Models}

\begin{itemize}   
\item Keras Models are composed of layers. 
\item The different layers constitute the Neural network. \item  The simple sequential composition models are called Sequential models. 
\item Sequential are added with \emph{model.add}. 

\item Sub-classing technique can be used to develop further complex models. To be discussed further.
\item Function API models are used to develop more flexible, complex models.  It uses graphs of layers.\\
\end{itemize}
\end{frame}

\begin{frame}{Elements of Keras}{Layers}
\begin{itemize}
\item Layers are next in the hierarchy of the Keras structure. 
\item They are input, hidden and output layer in neural network models. 

\item Most common available pre-defined layers are: 
\begin{itemize}
    \item \textbf{Convolutional}, \textbf{
    \item Pooling}, 
    \item \textbf{Recurrent} 
    \item \textbf{Core}.
\end{itemize}
\item In between the previous layers, \textbf{dropout} layers can be added to reduce overfitting.

\end{itemize}
\end{frame}
\begin{frame}{Elements of Keras}{Core}
\begin{itemize}
\item  Basic building blocks of any keras model architecture. 
\item They are built-n functions that support the Keras model ensuring its proper functioning. 
\item The modules used include: 
\begin{itemize}
    \item \emph{Activation functions} such as softmax, relu, etc, 
    \item \emph{loss function module} (mean square error, Poisson, likelihood,  mean absolute error etc.), 
    \item \emph{optimizer module} that uses optimizers such as adam, Stochastic gradient descent (SGD), etc. and \emph{regularizers} (L1 and L2 regularizers). 
\end{itemize}
\item These pre-defined modules  support the training of the Neural network models.
\end{itemize}
\end{frame}

\end{document}	

