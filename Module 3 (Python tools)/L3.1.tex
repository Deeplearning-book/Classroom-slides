\documentclass{beamer}
\setbeamertemplate{footline}[page number]
\date{}
\author{}
\institute{}

%%%%%%% Put these names back in the final version 
%\\Aswathy Rajendra Kurup\\Meenu Ajith}
%\institute{Department of Electrical and Computer Engineering\\The University of New Mexico}
\setbeamercovered{transparent}
\usepackage{setspace}
\usepackage{array}
\usepackage[T1]{fontenc}
\usepackage{graphicx}
\usepackage{amsmath}
\usepackage{amsfonts}
\usepackage{amssymb}
\usepackage{makeidx}
\usefonttheme{serif}
\usepackage{multirow}
\usepackage{booktabs} 
\usepackage{rotating}
\usepackage{color}
\usepackage{float}
\usepackage[latin1]{inputenc}
\usepackage[english]{babel}
\usepackage{amsmath}
\usepackage{amsfonts}
\usepackage{eurosym}
\usepackage{rotating}
\usepackage{multicol}
\usepackage{pythonhighlight}
\usepackage[normalem]{ulem}
\newcommand{\ba}{{\bf a}}
\newcommand{\bb}{{\bf b}}
\newcommand{\bc}{{\bf c}}
\newcommand{\bd}{{\bf d}}
\newcommand{\be}{{\bf e}}
\newcommand{\bbf}{{\bf f}}
\newcommand{\bg}{{\bf g}}
\newcommand{\bh}{{\bf h}}
\newcommand{\bi}{{\bf i}}
\newcommand{\bk}{{\bf k}}
\newcommand{\bl}{{\bf l}}
\newcommand{\bm}{{\bf m}}
\newcommand{\bn}{{\bf n}}
\newcommand{\bo}{{\bf o}}
\newcommand{\bp}{{\bf p}}
\newcommand{\bq}{{\bf q}}
\newcommand{\br}{{\bf r}}
\newcommand{\bs}{{\bf s}}
\newcommand{\bt}{{\bf t}}
\newcommand{\bu}{{\bf u}}
\newcommand{\bv}{{\bf v}}
\newcommand{\bw}{{\bf w}}
\newcommand{\bx}{{\bf x}}
\newcommand{\by}{{\bf y}}
\newcommand{\bz}{{\bf z}}

\newcommand{\bA}{{\bf A}}
\newcommand{\bB}{{\bf B}}
\newcommand{\bC}{{\bf C}}
\newcommand{\bE}{{\bf E}}
\newcommand{\bG}{{\bf G}}
\newcommand{\bH}{{\bf H}}
\newcommand{\bI}{{\bf I}}
\newcommand{\bK}{{\bf K}}
\newcommand{\bL}{{\bf L}}
\newcommand{\bM}{{\bf M}}
\newcommand{\bO}{{\bf O}}
\newcommand{\bQ}{{\bf Q}}
\newcommand{\bR}{{\bf R}}
\newcommand{\bS}{{\bf S}}
\newcommand{\bT}{{\bf T}}
\newcommand{\bV}{{\bf V}}
\newcommand{\bW}{{\bf W}}
\newcommand{\bX}{{\bf X}}
\newcommand{\bY}{{\bf Y}}
\newcommand{\bZ}{{\bf Z}}
\newcommand\uptocnt{\stackrel{\mathclap{\normalfont\mbox{c}}}{\propto}}
\newcommand{\bpt}{{\bf pt}}
\newcommand{\bpl}{{\bf pl}}
\newcommand{\bdp}{{\bf dp}}
\newcommand{\btemp}{{\bf temp}}

\newcommand{\bmu}{{\boldsymbol \mu}}
\newcommand{\bSigma}{{\boldsymbol \Sigma}}
\newcommand{\bsigma}{{\boldsymbol \sigma}}
\newcommand{\bvarPhi}{{\boldsymbol \varPhi}}
\newcommand{\bvarphi}{{\boldsymbol \varphi}}
\newcommand{\bPhi}{{\boldsymbol \Phi}}
\newcommand{\bdelta}{{\boldsymbol \delta}}
\newcommand{\bZero}{{\bf 0}}
\newcommand{\bOne}{{\bf 1}}
\newcommand{\balpha}{{\boldsymbol \alpha}}
\newcommand{\bAlpha}{{\boldsymbol A}}
\newcommand{\btheta}{{\boldsymbol \theta}}

\newcommand{\softmax}{\text{softmax}}
\newcommand{\diag}{\text{diag}}
\newcommand{\sinc}{\mathrm{sinc}}
\newcommand{\argmin}{\mathop{\mathrm{argmin}}}
\newcommand{\infl}{\eta}
\newcommand{\Ind}{\mathrm{I}}
\newcommand{\Real}{\mathbb R}
\newcommand{\Intg}{\mathbb Z}
\newcommand{\Complex}{\mathbb C}
\newcommand{\Natural}{\mathbb N}
\newcommand{\Fourier}[1]{\mathcal{F} \{#1\}}
%\newcommand{\ii}{\mathbbm{i}}
\newcommand{\bphi}{\boldsymbol{\mathit{\phi}}}

\newcommand{\hs}{\hspace{2pt}}
\newcommand{\sign}{\text{sign}}
\author{Manel Mart\'inez-Ram\'on\\Meenu Ajith\\Aswathy Rajendra Kurup}

\usetheme{Madrid}
\usecolortheme{beaver}
\usepackage{tikz}
\usetikzlibrary{fit,arrows,calc,positioning}
\usepackage{listings}
\usepackage{xcolor}
\usepackage{emerald} 
\usepackage[T1]{fontenc} 
\usepackage{verbatim}
\usepackage{graphicx}
\usepackage{epsfig}
\usepackage{psfrag}
\usepackage[english]{babel}
\usepackage{listings}
\usepackage{courier}
\usepackage{color}
 \usepackage{vwcol} 
 \usepackage[english]{babel} % To obtain English text with the blindtext package
\usepackage{blindtext}
\definecolor{codegreen}{rgb}{0,0.6,0}
\definecolor{codegray}{rgb}{0.5,0.5,0.5}
\definecolor{codepurple}{rgb}{0.58,0,0.82}
\definecolor{backcolour}{rgb}{0.95,0.95,0.92}

\lstdefinestyle{mystyle}{
  backgroundcolor=\color{backcolour},   commentstyle=\color{codegreen},
  keywordstyle=\color{magenta},
  numberstyle=\tiny\color{codegray},
  stringstyle=\color{codepurple},
  basicstyle=\ttfamily\footnotesize,
  breakatwhitespace=false,         
  breaklines=true,                 
  captionpos=b,                    
  keepspaces=true,                 
  numbers=left,                    
  numbersep=5pt,                  
  showspaces=false,                
  showstringspaces=false,
  showtabs=false,                  
  tabsize=2
}
\lstset{style=mystyle}

%% Stuff for movies

% %\newcommand{\bt}{{\bf t}}
% \newcommand{\br}{{\bf r}}
% \newcommand{\bs}{{\bf s}}
% \newcommand{\by}{{\bf y}}
% \newcommand{\bz}{{\bf z}}
% \newcommand{\bx}{{\bf x}}
% \newcommand{\bw}{{\bf w}}
% \newcommand{\be}{{\bf e}}
% \newcommand{\bbf}{{\bf f}}
% \newcommand{\bb}{{\bf b}}
% \newcommand{\bd}{{\bf d}}
% \newcommand{\bA}{{\bf A}}
% \newcommand{\bB}{{\bf B}}
% \newcommand{\bL}{{\bf L}}
% \newcommand{\bM}{{\bf M}}

% \newcommand{\bC}{{\bf C}}
% \newcommand{\bI}{{\bf I}}
% \newcommand{\bK}{{\bf K}}
% \newcommand{\bk}{{\bf k}}
% \newcommand{\bT}{{\bf T}}
% \newcommand{\bV}{{\bf V}}
% \newcommand{\bW}{{\bf W}}
% \newcommand{\bX}{{\bf X}}
% \newcommand{\bY}{{\bf Y}}
% \newcommand{\bZ}{{\bf Z}}
% \newcommand{\bm}{{\bf m}}
% \newcommand{\bpt}{{\bf pt}}
% \newcommand{\bpl}{{\bf pl}}
% \newcommand{\bdp}{{\bf dp}}
% \newcommand{\btemp}{{\bf temp}}
% \newcommand{\bl}{{\bf l}}
% \newcommand{\bu}{{\bf u}}
% \newcommand{\bmu}{{\boldsymbol \mu}}
% \newcommand{\bSigma}{{\boldsymbol \Sigma}}
% \newcommand{\bLambda}{{\boldsymbol \Lambda}}

% \newcommand{\bsigma}{{\boldsymbol \sigma}}
% \newcommand{\bvarphi}{{\boldsymbol \varPhi}}
% \newcommand{\btheta}{{\boldsymbol \theta}}
% \newcommand{\bZero}{{\bf 0}}
% \newcommand{\balpha}{{\boldsymbol \alpha}}
% \newcommand{\bpi}{{\boldsymbol \pi}}
% \newcommand{\bxi}{{\boldsymbol \xi}}
% \newcommand{\bdelta}{{\boldsymbol \delta}}
\lstset{
	language=Python,
	basicstyle=\footnotesize\ttfamily\color{black},
	commentstyle = \footnotesize\ttfamily\color{red},
	keywordstyle=\footnotesize\ttfamily\color{blue},
	stringstyle=\footnotesize\ttfamily\color{black},
%	columns=fixed,
%	numbers=left,    
	numberstyle=\tiny,
	stepnumber=1,
	numbersep=5pt,
	tabsize=1,
	extendedchars=true,
	breaklines=true,            
	frame=b,         
	showspaces=false,
	showtabs=true,
	xleftmargin=6pt,
	framexleftmargin=6pt,
	framexrightmargin=2pt,
	framexbottommargin=4pt,
	showstringspaces=false      
}

\lstloadlanguages{
         Python
}

%\graphicspath{ {./images/} }  % Figures path - used in graphicx

%\selectcolormodel{cmyk}

\mode<presentation>

\newcommand{\dred}{darkred!90!black}
\newcommand{\written}{\ECFJD\textcolor{cyan!50!white}}
\newcommand{\hlight}{\textcolor{\dred}}
\newcommand{\Ex}{\textcolor{\dred}{Ex. }}

% remove navigation symbols in full screen mode
\setbeamertemplate{navigation symbols}{}  
\setbeamertemplate{blocks}[rounded][shadow=false]
\setbeamercolor{note page}{fg=black}

\setbeamercolor{title}{fg=\dred}
\setbeamercolor{frametitle}{fg=white}
\setbeamercolor{frametitle}{bg=\dred}
\setbeamercolor{structure}{fg=black,bg=white}
\setbeamercolor{background canvas}{bg=white,fg=black}
\setbeamercolor{normal text}{fg=black,bg=white}
\setbeamercolor{item}{fg=red!80!black,bg=white!}
\addtobeamertemplate{block begin}{\setbeamercolor{block title}{fg=white,bg=\dred}
\setbeamercolor{block body}{fg=white,bg=gray}}{}


 
\title{3. Deep learning tools}
\subtitle{3.1 Phython: an overview}

\addtobeamertemplate{frametitle}{}


\begin{document}

\maketitle

\begin{frame}{Introduction}
\begin{multicols}{2}
\begin{itemize}
    \item First released in 1991. 
    \item Syntax designed for code readability.
    \item Extendable: Interacts with C/C++.
    \item Large range if libraries. Compatible with all OS. 
   
\end{itemize}
\columnbreak
\includegraphics[scale=0.3]{Module 3 (Python tools)/pics/python.png} 
\begin{itemize}
\item Downloads and tutorials in python.org.
\end{itemize}
\end{multicols}
\end{frame}


\begin{frame}[fragile]{Variables}
\begin{itemize}
    \item Lists and tuples: 
\begin{lstlisting}
#this is a list
list=[2,4,'hello',3]
#this is a tuple
tuple=(2,4,'hello',3)
\end{lstlisting}

\item Strings, and sets:  
    
\begin{lstlisting}
#this is a string 
string='hello'
#This is a proper set
set={0,1, (2,4,'hello',3)}
#This set will not work
set={0,1, [2,4,'hello',3]}
\end{lstlisting}
\item Tuples and sets are immutable, so they cannot contain lists.
\end{itemize}
\end{frame}

\begin{frame}[fragile]{Variables}
\begin{itemize}
\item Collection of elements represented as key-value pairs. Keys are immutable data types.

\begin{lstlisting}
diction = {'number':23, 'name':'Tom','age': 16} 
#defining a dictionary
the_diction = {} #empty dictionary
\end{lstlisting}

\item To access each element of a dictionary, the keys are used along with the square brackets.

\begin{lstlisting}
print(diction['age']) #output is 16.
diction['name'] = 'Harry'
print(diction)
#outputs is {'number':23,'name':'Harry','age': 16}
del diction['age']
print(diction) #outputs is {'number':23,'name':'Harry'}.
print(diction.pop('name')) 
#outputs 'Harry' and removes it from the dictionary.

\end{lstlisting}
\end{itemize}
    
\end{frame}


\begin{frame}[fragile]{Conditional Statements}

\begin{lstlisting}{}

a = 1
if a == 2:
   print('Even prime number')
elif a == 3:
    print('Odd prime number')
else:
   print('Neither prime nor composite number')
\end{lstlisting}
\end{frame}


\begin{frame}[fragile]{Loops}

\begin{lstlisting}
for j in range(6,10):
     print(j) #outputs the numbers 6,7,8 and 9.
\end{lstlisting}


\begin{lstlisting}
num = [2,4,6,8,10]
for i in num:
   if i == 6:
       break
    print(i) #outputs numbers 2 and 4.
\end{lstlisting}

\begin{lstlisting}{}
seq = ['h','e','l','l','o']
for j in seq:
     print(j, end = '')
#prints "hello"
#command "pass" is used as a placeholder for future code.
\end{lstlisting}

\end{frame}

\begin{frame}[fragile]{Functions}
Functions are the basic way to write reusable code.
\begin{lstlisting}
def addition(a,b): #function name and parameters.
     out = a+b #function body.
     return  out  #return value of the function.
#Here the function definition ends.
#Note the indentation change from next line.
a = 10
b = 40
out = addition(a,b)
print(out) #outputs the number 50. 
\end{lstlisting}

\end{frame}

\begin{frame}[fragile]{Objects and classes}

\begin{lstlisting}{}
class Smartphone: #Defining the object.
 def __init__(self, brand,storage):
  self.brand=brand \#Defines the object properties.
  self.storage=storage 
 def capacity(self): \#Defining the object behaviour.
  if self.storage >= 512:
   return 'Large Storage'
  elif (self.storage < 512) & (self.storage > 128):
   return 'Average Storage'
  else:
   return 'Low Storage'

phone1 = Smartphone('EyePhone',256)
space = phone1.capacity()
print(phone1.brand) #outputs 'EyePhone'.
print(phone1.storage) #outputs 256.
print(phone1.brand, ":",space) 
#outputs EyePhone : Average Storage.
\end{lstlisting}
\end{frame}


\end{document}	